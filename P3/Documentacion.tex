\documentclass[12pt, letterpaper]{article}
\usepackage{graphicx}
\usepackage{ragged2e}
\usepackage{enumitem}
\usepackage[T1]{fontenc}
\usepackage[spanish]{babel}
\graphicspath{{../cap}}

\title{Cuaderno de prácticas. \\Práctica 3. Greedy y Programacion Dinamica.}
\author{Enrique Pinazo Moreno}
\date{\today}

\begin{document}
  \maketitle
  \tableofcontents
  \newpage %Pagina 1

  \section{Algoritmo Greedy (Algoritmo Voraz).}
    \subsection{Diseño del Algoritmo Voraz.}
      \begin{itemize}
        \item \textbf{Componentes del Diseño Greedy.}
          \subitem\textbf{- Respuesta:}
             
        \item \textbf{Criterio Heuristico y Ejemplo.}
          \subitem\textbf{- Respuesta:}
            Para resolver el problema generado por la cláusula default(none), es necesario especificar explícitamente qué variables son privadas y cuáles son compartidas. 
            En el código original, se puede hacer esto utilizando las cláusulas private y shared en la directiva parallel. 
            \newline En este caso, hemos especificado que las variables a, b y c son compartidas entre los hilos, mientras que la variable i es privada para cada hilo.
            
            \newpage %Pagina 2
      \end{itemize}
    
    \newpage %Pagina 3

\end{document}